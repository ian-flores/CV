% YAAC Another Awesome CV LaTeX Template
%
% This template has been downloaded from:
% https://github.com/darwiin/yaac-another-awesome-cv
%
% Author:
% Christophe Roger
%
% Template license:
% CC BY-SA 4.0 (https://creativecommons.org/licenses/by-sa/4.0/)
%Section: Work Experience at the top
\sectionTitle{Professional Experience}{\faSuitcase}
%\renewcommand{\labelitemi}{$\bullet$}
\begin{experiences}
  \experience
    {Present}   {Graduate Academic Assistant \& Teaching Assistant}{University of British Columbia}{Canada}
    {January 2019} {
                      \begin{itemize}
                        \item Develop educational material for the Introduction to Data Science course (DSCI 100)
                        \item Contribute to the development of new reproducible tools for education
                        \item Course materials available here: \href{https://github.com/ubc-dsci/dsci-100}{https://github.com/ubc-dsci/dsci-100}.
                      \end{itemize}
                    }
                    {R, Tidyverse, Python, Jupyter, Docker}
  \emptySeparator
  
  \experience
    {August 2018}   {Information System Consultant}{Puerto Rico Institute of Statistics}{Puerto Rico}
    {March 2018} {
                      \begin{itemize}
                        \item Developed an open source application to visualize financial data from the Government of Puerto Rico.                         
                        \item Completed the data services infrastructure of the Puerto Rico Violent Death Reporting System.
                        \item The code for this application is available here: \newline
                        \href{https://github.com/ian-flores/TransparenciaFinanciera}{https://github.com/ian-flores/TransparenciaFinanciera}
                      \end{itemize}
                    }
                    {R, Tidyverse, RBokeh, Python, Jupyter, Docker, CouchDB, React}
  \emptySeparator
  
  \experience
    {August 2018}   {Undergraduate Researcher - Computer Science}{University of Puerto Rico}{Puerto Rico}
    {July 2017} {
                      \begin{itemize}
                        \item Applied Convolutional Neural Networks to classify animal callings in audio recordings.    
                        \item Devised data augmentation techniques for images of the callings to lower the validation \newline error of classification algorithms.
                        \item The code for this research project is available here: \newline
                        \href{https://github.com/ian-flores/Deep-Learning-Species-Identification}{https://github.com/ian-flores/Deep-Learning-Species-Identification}
                      \end{itemize}
                    }
                    {Python, TensorFlow}
  \emptySeparator
  
  \experience
    {July 2016}   {Eco-Informatics Intern}{National Ecological Observatory Network}{United States}
    {May 2016} {
                      \begin{itemize}
                        \item Developed an open source application to visualize the spatial distribution and temporal patterns of animals in the continental United States.
                        \item Investigated the use of visualizations to make science more accessible to non-scientific users.
                        \item The code for this application is available here: \newline
                        \href{https://github.com/ian-flores/NEON-2016-Internship-DPS}{https://github.com/ian-flores/NEON-2016-Internship-DPS}
                      \end{itemize}
                    }
                    {R, ggplot2}
  \emptySeparator
  
  \experience
    {May 2016}   {Undergraduate Researcher - Biology}{University of Puerto Rico}{Puerto Rico}
    {July 2015} {
                      \begin{itemize}
                        \item Analyzed the spatial distribution of malaria in lizards in a secondary forest in Puerto Rico.
                        \item Implemented regression methods to explain the distribution of the disease.
                        \item The code for this research project is available here: \newline
                        \href{https://github.com/ian-flores/Spatial-Heterogeneity}{https://github.com/ian-flores/Spatial-Heterogeneity}
                      \end{itemize}
                    }
                    {R}
  \emptySeparator
  
\end{experiences}
