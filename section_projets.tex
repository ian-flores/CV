% Awesome Source CV LaTeX Template
%
% This template has been downloaded from:
% https://github.com/darwiin/awesome-neue-latex-cv
%
% Author:
% Christophe Roger
%
% Template license:
% CC BY-SA 4.0 (https://creativecommons.org/licenses/by-sa/4.0/)

%Section: Project
\sectionTitle{Projects}{\faLaptop}

\begin{projects}
    %\project
    %{Project Name}
    %{Project Dates}
    %{\github{GITHUB PAGE}
    %\website{WEBSITE_LINK}{WEBSITE_NAME}}
    %{Text describing project}
    %{tag1, tag2, tag3}
    
    \project
    {What the Git is going on here?}
    {April 2019 - June 2019}
    {Capstone project with RStudio}
    {\begin{itemize}
        \item This project aims to understand how people are currently using GitHub, with the eventual goal of building an easy-to-use alternative to Git.
    \end{itemize}}
    {Python, NetworkX, GitHub, Git, Google Big Query, SQL}
    
    \project
    {ColourblindR: An R package that creates colourblind friendly themes}
    {February 2019}
    {\github{ubc-mds/ColourblindR}}
    {\begin{itemize}
        \item Brand new theme package implemented for ggplot2 to optimize graphs into a format interpretable by people with colourblindness.
        \item The package is inspired by the fact that people without knowledge about this condition don’t know how to make their graphs accesible.
    \end{itemize}}
    {R, ggplot2, Travis CI}

    \project
    {Vancouver Property Value Analysis}{January 2019}
    {\github{UBC-MDS/Vancouver-Property-Value-Analysis}}
    {\begin{itemize}
    \item Built a data visualization application that allows users to visually explore residential property value and socio-economic data geographically mapped to Vancouver neighbourhoods.
    \end{itemize}}
    {R, Leaflet, Shiny}

%    \project
%    {\newline San Francisco Crime Resolution Model}{December 2018}
%    {\github{UBC-MDS/San\_Francisco\_Crime\_Resolution\_Model}}
%    {\begin{itemize}
%    \item Implemented a decision tree to classify San Francisco crime data based on whether a person would have been processed or not into the justice system.
%    \item Achieved an 83\% accuracy on validation data with the decision tree algorithm.
%    \end{itemize}}
%    {Python, scikit-learn, pandas}

    \project
    {\newline Hurricane Maria Mortality Analysis}{October 2018}
    {\github{ian-flores/Hurricane\_Maria\_Mortality\_Analysis}}
    {\begin{itemize}
    \item Estimated mortality rates for urban and rural areas in Puerto Rico after Hurricane Maria using Bayesian methods.
    \item Analyzed the spatial distribution of deaths in Puerto Rico after Hurricane Maria using US Census data.
    \end{itemize}}
    {Python, PyMC3, geopandas}

\end{projects}